\begin{figure*}
\centering
\begin{tabular}{cc}
\begin{lstlisting}
int logicalValue(int t) {
  if (!(currentTime - t >= 100)) {
    return old;
  } else {
    int val = 0;
    for (int i = 0 ; i < data.length; i++)
      val = val + data[i];
    old = val;
    return val;
  }
}
\end{lstlisting}
&
\begin{lstlisting}
const int THRESHOLD = 100;
int logicalValue(int t) {
  int elapsed = currentTime - t;
  int val = 0;
  if (elapsed < THRESHOLD) { 
    val = 1; 
  } else {
    for (int i = 0 ; i < data.length; i++) 
      val = val + data[i];
    old = val;      
  }
  return val;
}
\end{lstlisting}
\end{tabular}
\caption{Two versions of the \scode{logicalValue()} procedure taken from~\cite{DEP:FSE08}}
\figlabel{DSEExample}
\end{figure*} 