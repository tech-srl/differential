\section{Appendix}\label{Se:Appendix}


\subsection{Algorithm 2 : Convert $P$ To Guarded Instruction Format} \label{Se:GuardedAlg}
The algorithm is constructive i.e. it takes a procedure $P$ and outputs the new lines for a guarded version of $P$. The original $P$ is not part of the output.
\begin{itemize}
\item \emph{Stage 0:} Output $P$'s signature.
\item \emph{Stage 1:} Convert all \scode{while} constructs to \scode{if} and \scode{goto} constructs.
\item \emph{Stage 2:} For each non branch instruction $I$:
\subitem If $I$ is a declaration, output it under a new block.
\subitem Otherwise collect all branch conditions $C$ under which $I$ executes. Produce the code line: \scode{if ($\bigwedge{C}$) I;}
\end{itemize}

We take a small but sufficient example to demonstrate the algorithm:
\begin{figure}[H]
\begin{lstlisting}
void foo(char Out[], int n) {
    char In[42];
    int i;
    if (n >= 42) n = 41; (*@\label{Ln:ncheck}@*)
    while (i < n) {
        int j = i + 1;
        In[i] = Out[j];
        i++;
    }
}
\end{lstlisting}
Stage 1 Output:
\begin{lstlisting}
void foo(char Out[], int n) {
    char In[42];
    int i;
    if (n >= 42) n = 41;
l:  if (i < n) {
        int j = i + 1;
        In[i] = Out[j];
        i++;
        goto l;
    }
}
\end{lstlisting}
Stage 2 Output:
\begin{lstlisting}
void foo(char Out[], int n) {
    char In[42];
    int i;
    if (n >= 42) n = 41;
    {
        int j;
l:      if (i < n) j = i + 1;
        if (i < n) In[i] = Out[j];
        if (i < n) i++;
        if (i < n) goto l;
    }
}
\end{lstlisting} \caption{Algorithm 1 application example}
\end{figure}

Note that as required, the loop is implemented in means of branches and goto and that the branches do not exceed the 1 level of nesting allowed.

\subsubsection{Algorithm 2 Assumptions}
\begin{enumerate}
\item We only deal with \scode{while} loop statements.
\item Logical statement (branch conditions) do not have side-effects i.e. they do not change variable values.
\end{enumerate}


\subsection{Algorithm 1 Example} \label{SubSec:UnionAlgEx}

We run algorithm 1 on the previous example with the following patch to \lnref{ncheck}:\\
{\color{Gray}\scode{- if (n >= 42) n = 41;}} \\
{\color{Blue}\scode{+ if (n >= 42) n = 40;}}

We start off from the output of algorithm 2 i.e. a "guarded" program that performs simple buffer handling:
\begin{figure}[H]
\begin{lstlisting}
void foo(char Out[], int n) {
    char In[42];
    int i;
    if (n >= 42) n = 41;

    {
        int j;
l:      if (i < n) j = i + 1;
        if (i < n) In[i] = Out[j];
        if (i < n) i++;
        if (i < n) goto l;
    }

}
\end{lstlisting}
The result:
\begin{lstlisting}
void foo(char Out[], int n, char Out'[], int n') {
    char In[42], In'[42];
    int i, i';

    if (n >= 42) n = 41;
    if (n' >= 42) n' = 40;

    {
        int j,j';
l:      if (i < n) j = i + 1;
l':     if (i' < n') j' = j' + 1;
        if (i < n) In[i] = Out[j];
        if (i' < n') In'[i'] = Out'[j'];
        if (i < n) i++;
        if (i' < n') i'++;
        if (i < n) goto l;
        if (i' < n') goto l';
    }
}
\end{lstlisting} \caption{Algorithm 1 Application Example}
\end{figure}

