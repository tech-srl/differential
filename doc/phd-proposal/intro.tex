\section{Introduction} \seclabel{Intro}

We present a novel technique for identifying and describing semantic change in programs while relying on abstract interpretation and program analysis methods~\cite{CousotCousot77}. We define the problem of \emph{semantics differencing} as identifying observable difference in behavior for executions originating from \emph{the same input}. The notion of observable difference can be trivially defined as difference in output values in the final state~\cite{DwyerElbaumPerson08,GodlinStrichman09,EnglerRamos11,HawblitzelKawaguchiLahiriRebelo12}, but our research extends this notions to detect other important behavioral differences such as array access patterns, assertion failures, etc.

%Understanding the semantic difference between two versions of a program is invaluable in the process of software development. A developer making changes to a program is often interested in answering questions like:
%\begin{inparaenum}[(i)]
%\item did the patch add/remove the desired functionality?
%\item does the patch introduce other, \emph{unexpected}, behaviors?
%\item which regression tests should be run?
%\end{inparaenum}
%Answering these questions manually is difficult and time consuming.

Differential analysis has received attention in classical work (e.g.,~\cite{Horwitz90,Horwitz89,JacksonLadd94}) and recently been the subject of research in program analysis~\cite{DwyerElbaumPerson08,GodlinStrichman09,EnglerRamos11,HawblitzelKawaguchiLahiriRebelo12}. Hoare~\cite{Hoare69} defined program equivalence and differencing as part of the ``grand challenge'' to the verification community and recently suggested~\cite{HoareLahiriVaswani10} several applications where static analysis can be easily adopted as an effective solution for tasks such as equivalence checking, semantic diffing, differential contract checking, patch debugging, etc. Specifying semantic difference between program versions has seen growing interest for various applications ranging from testing of concurrent programs~\cite{ChakiGurfinkelStrichman12}, understanding software upgrades~\cite{JinOrsoXie10}, to automatic generation of security exploits~\cite{BrumleyPoosankamSongZheng08}.

In contrast to existing recent techniques for semantic differencing, that rely on variants of (bounded) symbolic execution and SMT solving~\cite{DwyerElbaumPerson08,GodlinStrichman09,EnglerRamos11,HawblitzelKawaguchiLahiriRebelo12}, our technique employs abstract interpretation for identifying difference. This means that our approach can provide a sound characterization of program differences. Further, our approach has the potential to scale better existing techniques by using abstractions of varying precision (and cost). Using abstract interpretation for differential analysis means reasoning over two programs. One of the challenges in applying abstract interpretation is the fact that equality of two abstract program states does not entail concrete equality. Furthermore, defining a useful notion of equality is a challenge on its own right. 
We address these challenges by presenting:
\begin{itemize}
\item a method for abstract interpretation of a pair of programs $(P,P')$ for \emph{sound} semantic equivalence and differencing by abstracting direct relationships between $(P,P')$ variables in a partially disjunctive domain. We describe a partitioning technique for state reduction and scaling. We define a widening operator for abstracting unbounded paths in our domain.
\item a new technique for syntactically interleaving a pair of programs $(P,P')$ for the creation of a \emph{correlating program} $P \bowtie P'$ which contains the semantics of both programs. We propose an analysis over the program for characterizing program equivalence and difference, based on the aforementioned abstraction, given the properties of the correlating program which aligns $(P,P')$ executions.
\end{itemize}

We have implemented our approach in a tool based on the \sname{LLVM} compiler infrastructure and the \sname{APRON} numerical abstract domain library, and evaluated it using select patches from open-source software including GNU core utilities, Mozilla Firefox, and the Linux Kernel. Our evaluation shows that the tool often manages to establish equivalence, reports useful approximation of semantic differences when differences exists, and reports only a few false differences.

We plan further research for refining and adjusting our technique to achieve better results on a wider range of programs by extending our analysis to be inter-procedural, using several abstract domains to capture differences in more complex data structures (pointers, heaps) and differencing concurrent programs.

We do not limit ourselves purely to abstract interpretation techniques and one future direction of our work will include applying existing dynamic analysis and symbolic execution techniques on correlated programs. The interleaved structure offered by the correlating program holds great potential as it alleviates the need to "delay" differencing until both programs have been analyzed completely (this will be further explained in \secref{Future}). Another major research direction will different compositions of the correlating program that will allow for better differencing. Specifically, an equivalence-guided correlation refinement technique can be used i.e. the way the programs are interleaved will be directed by the analysis according to equivalence criteria.

As our results so far have limited use as they basically offer an abstraction of program difference at a certain program location, we also plan to produce techniques for using our results towards practical applications such as regression test generation, patch installation debugging and synthesis of attacks from a patch.

