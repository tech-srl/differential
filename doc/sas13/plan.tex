\section{Research Plan}

First we plan on adding a more complex abstract domain for our analysis and working on producing input from differentiation data. We also want to address the analysis of more complex data structures (i.e. pointers) for widening the range of programs we can process. Finally we will offer techniques that prevent the synthesis of attacks based on differential analysis and examine how it can be used for other applications.

\para{Expanding The Analysis Abstract Domains.} We wish to incorporate a linear inequalities abstract domain into the CLang \cite{CLang} infrastructure for better analysis. We plan on using the APRON \cite{DBLP:conf/cav/JeannetM09} library and believe that incorporation of the two will benefit many future work.

\para{Producing Input.} Calculating the differentiation does not hold by itself. We will produce (whenever possible) explicit variable input data for the analyzed procedure. This will of course require careful tuning of the analysis.

\para{Analyzing Complex Data Structures.} Analyzing integer variable values is sufficient for only a subset of the programs. In order to be more relevant to a wider range of programs we will propose and examine methods for analyzing pointers, lists and other complex data structure and produce useful differential data for them.

\para{Mitigating Patch-Based Exploitation.} Learning how differential analysis can be used to deduce exploitable input for programs is key in mitigating attacks on un-patched programs. After clearly understanding how a patch can be studied to create a malicious input, we will know how to design the patch to be more resilient to this attacks.

\para{Applying Differential Analysis Towards Other Ends.} Differential analysis has other applications such as equivalence checking, regression test pruning, patch installation debugging etc. We will explore how our results could be used in other ways and inspire more research.
